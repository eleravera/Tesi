Pixel detectors, members of the semiconductor detector family, have fastely been used since the first 
accelerator experiments for energy and position measurement. 
Because of their dimension (today $\sim$ 30 $\mu m$ or even better) and their spatial resolution 
\cite{einstein}
($\sim$ 5-10 $\mu m$), with the availability of technology in 1980s they proved to be perfectly suitable 
for vertex detector in the inner layer of the detector.\\
Hybrid pixel currently constitute the state-of-art for large scale pixel detector but experiments began to look at monolitic active pixels (MAPS) as perspective for their future upgrades; some of them, as ALICE or 
(), have already installed them. \\

Today almost every high energy physics experiment employs semiconductor detector and 
lo sviluppo tecnologico e la ricerca sono indirizzate nel trovare modi per ridurre il material budget e
avere una lettura sempre più veloce: due dei principali limiti attuali per questi detector. \\

Per gli acceleratori la richieste sono molto stringenti soprattutto per il futuro ad es HL LHC in termini di radiation hardness (expected in 5 anni 500 Mrad e NIEL di 10 alla 16), efficiency e occupancy (efficienza alta dopo tanta radiazione e noise occupancy bassa), time resolution (bunch crossing 40 Mhz), material budget e power consumption (material budget below 2per cento e power consumption 500 mW/cm2)

Cos'è la flash therapy\\
Problemi sperimentali della flah therapy\\
Quali sono i limiti sperimentali richiesti per un dosimetro.\\
PERCHè i pixel potrebbero essere buoni per la dosimetria.\\


\begin{comment}
\begin{figure}
    \begin{minipage}{0.48\textwidth}
      \centering
      \includegraphics[width=.8\linewidth]{figures/MAPS_scheme.png}
    \end{minipage}
    \begin{minipage}{0.48\textwidth}
      \centering
      \includegraphics[width=1.\linewidth]{figures/DMAPS_scheme.png}
    \end{minipage}
    \caption{Concept cross section of MAPS and DMAPS pixel}
    \label{fig:MAPS_DMAPS_scheme}
 \end{figure}


 \subsubsection{Position measurement resolution}
Depending on the type of signal reading the spatial resolution is 
$\sigma_x = \frac{p}{\sqrt{12}}$ where $p$ is the pitch between pixels, or even better if other 
analogica information, as the charge, are read and capacitive charge division method is applied.
\end{comment}
