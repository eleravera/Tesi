La radioterapia si usa nel 60 per cento dei pazienti, sia come cura che come trattamento palliativo. Si associa spesso ad altre cure e si può fare prima/durante/dopo un intervento.\\

Si può fare in modi diversi: da dentro (brachytherapy) oppure da fuori (quella standard). Un requisito importante è la delinazione del target (non vuoi rischiare di beccare i tessuti sani), per cui prima tipicamente si fanno esami di imaginig del tumore. TIpicamente anche gli acceleratori stessi per la terapia sono provvisti di radiografia.

Un problema dei fotoni ad esempio è che il loro rilascio di dose è lineare, per cui danneggi anche i tessuti sani. Il problema dei protoni invece è che hanno un picco troppo strtto per cui non puoi coprire grosse zone e sorpattuto se sbagli rischi davvero di danneggiare moooolto i tessuti sani.\\

\section{Cell survival curves}
Curva di efficacia del trattamento in funzione della dose:
\begin{equation}
    \frac{S(D)}{S(0)}=e^{-F(D)}
\end{equation} 
dove F(D)   
\begin{equation}
    F(D) = \alpha D + \beta D^2
\end{equation} 
dove $\alpha$ e $\beta$ rappresentano due tipi di danno diversi: coefficients, experimentally determined, characterizing the
radiation response of cells. In particularly, alpha represents the rate of cell killing
by single ionizing events, while beta indicates the maximal rate of cell killing by
double hits observed when the repair mechanisms do not activate during the
radiation exposure.
Si ottiene una curva di sopravvivenza dove si vede la possibilità delle cellule di autoripararsi. A basse dosi infatti le cellule possono ripararsi.\\

Per introdurre l'effetto FLASH instroduco prima la therapeutic window.  \\

TCP è la tumor control Probability che indica la probabilità delle cellule del tumore di essere uccise dopo una certa dose (con in riferimento a dose in acqua)\\
Se una media di $\mu(D)$ di cellule di tumore are killed con una dose D, la probabilità che n cellule sopravvivono è data da $P(n|\mu)$ poisson:
\begin{equation}
    P(n|\mu) = \frac{\mu(D)^ne^{-\mu(D)}}{n!}
\end{equation}    

\begin{equation}
    TPC(D) = P(n=0|\mu(D))= e^{-\mu(D)}
\end{equation} 
D'altra parte hai una probabilità di fare danno su normal tissue NTCP Normal Tissue Complication Probability, che rappresenta il problema principale e che limita la massima radiazione erogabile\\
Una scelta bilanciata si applica guardando a questi due fattori; si usa il therapeutic index definito come TCP/NTCP.\\
La cosa ottimale è ampliare la finestra del therapeutic ratio.\\


CONV-RT 0.01-5 Gy/min. A typical RT regime today consists of daily franctions of 1.5 to 3 Gy given over several weeks.\\
Nell Intra operative radiation therapy (IORT), where they reach values respectively about
20 and 100 times greater than those of conventional radiation therapy.

FLASH vuole ultrahigh mean dose-rate (maggione di 40 Gy/s) in modo da ridurere anche il trattamento a meno di un secondo. \\



\section{FLASH effect}
Ci sono due effetti che affect the flsh effect and la sua applicabilità: Dose rate effect e oxygen\\

Cellule che esibiscono hypoxia (cioè cellule che non hanno ossigeno sono radioresistenti); al contrario normoxia e physoxia non lo sono.
la presenza di ossigeno rende la curva steeper indicando che lo stesso danno si raggiunge a livelli di dose più bassi rispetto al caso senza ossigeno.\\
FIGURA con una curva a confronto con e senza ossigeno.\\
Typically, the OER is in the order of 2.5–3.5 for most cellular systems


Quindi si vogliono sfruttare questi effetti per diminuire la tossicità sui tessuti sani\\




