There always was a tight relation between the development of cameras and pixel detectors since 1969, when the idea of CCDs, thanks to whom Boyle and Smith were awarded the Nobel Prize in Physics in 2009, revolutionized photography allowing light to be captured electronically instead of on film. 
Even though the CMOS technology was already known when CCDs spread, the costs of productions were too high to allow the diffusion of these sensors for which needed to wait untill 1990s. From that period on, the fast diffusion of CMOS was mainly due to the less cost than CCD, and the less power required for supply. 

The principal use cases of pixel detectors are particle tracking and imaging: in the former case individual charged particles have to be identified, in the latter instead an image is obtained by the usually un-triggered accumulation of the impinging radiation. 
Also the demands on detectors performance depends on their usage, in particular tracking requires high spatial resolution, fast readout and radiation hardness. 

For scientific imaging, instead, the applications vary from atrophysics and medical imaging to studies of protein dynamics, \red{altro?} and art authentication, for example. 
The counting mode represents the principal imaging tecnique, a direct 

\section{Tracking in HEP}
    Historically, the first pixel detector employed in particle physics was a CCD: it was installed in the spectrometer at the CERN’s Super Proton Synchrotron (SPS) by the ACCMOR Collaboration (Amsterdam, CERN, Cracow, Munich, Oxford, RAL) at mid 1980s, with the pourpose of studing the recently-discovered charm particles.
    The second famous usage of CCDs took place at SLAC in the Large Detector (SLD) during the two years 1996-98. From that period on particle tracking in experiments have been transformed radically: it was mandatory for HEP experiments to build a inner vertex detector. 
    In 1991, the more demanding environments led to the development of hybrid pixel detectors: a dedicated collaboration, RD19, was established at CERN with the specific goal to define a semiconductor micropattern detector with an incorporated signal processing at a microscopic level. 
    In those years a wide set of prototypes of hybrid pixel has been manufactured; among the greatest productions a mention goes to the huge ATLAS and CMS vertex detectors. 
    From the middle of 2013 a second collaboration, RD 53, has been established with the new goal to find a pixel detector suitable for phase II future upgrades of those experiments. Even if the collaboration is specifically focused on design of hybrid pixel readout chips (aiming to \SI{65}{nm} tecnique so that the electronics fits within the pixel area), also other options have been taken in account and many test have been done on MAPS for example. Requirements imposed by HL-LHC will become tigher in time: for example, a dose and radiation of \SI{5}{Mrad} and \si{10 {16}}{NIEL} are exepcted after 5 years of operation. Time resolution, material budget and power consumption are also issues for the upgrade: a time resolution better than \SI{25}{ns} for a bunch crossing frequency of \SI{40}{MHz}, a material budget lower than 2\% and a power consuption lower than  \SI{500}{mW/cm\squared} are required. 

    Amidst the solutions proposed 3D silicon detector, invented by Sherwood Parker in 1995, and MAPS are the most promising. In 3D sensors the electrode is a narrow column of n-type implanted vertically across the bulk instead of being implanted on the wafer's surface. 
    The charge produced by the impinging particle is then drifted transversally within the pixel, and, as the mean path between two electrode can be soufficent low, the trap probability is not an issue. 
    3D pixels have been already proved in ATLAS tracker \red{qualcosa? tipo anno e rif a caso.} 
    Even if 3D detector are adequately radiation hard, MAPS architecture looked very promising from the beginning: they overcome both the CCDs long reading time and the hybrid problems (I have already explained in section \ref{sec:} the benefits of MAPS). 
    Experiments such as ALICE at LHC and STAR at RHIC have already introduced the CMOS MAPS technology in their detectors. ALICE Tracking System (ITS2), upgraded during the LHC long shut down in 2019-20, was the first large-area ($\sim$10 \si{m\squared} covered by 2.5 Gpixels) silicon vertex detector based on CMOS MAPS.

    \subsection{Hybrid pixels at LHC: ATLAS, CMS and LHC-b}
        \textbf{ATLAS}\\
        ATLAS is one of two general-purpose detectors at the LHC and has the largest volume detector ever constructed for a particle
        collider (\SI{46}{m} long and \SI{25}{m} in diameter).  
        The inner detector consists of three different systems all immersed in a magnetic field parallel to the beam axis. The main components of the Inner Detector are: the pixel, the micro-strips and transition radiation trackers. 92 million pixels are divided in 4 barrel layers and 3 disks in each end-cap region, covering a total area of \SI{1.9}{m\squared} and having a \SI{15}{kW} of power consumption.

        As stated by the ATLAS collaboration the pixel detector is exposed by an extreme particle flux: "By the end of Run 3\footnote{Run 3 start in June 2022}, the number of particles that will have hit the innermost pixel layers will be comparable to the number it would receive if it were placed only a few kilometres from the Sun during a solar flare". And the particle density will increase even more with HL-LHC. 
        The most ambitious goal is employ a MAPS-based detector for the inner-layer barrels, and for this reason the RD53 collaboration is performing many test on MAPS prototypes, as Monopix of which I will talk about in section \ref{sec:}.
        
        Up to now this possibility will be eventualy implemented during the second phase of the HL-LHC era, as at the start of high-luminosity operation the selected option is the hybrid one. The sensor, which is not selected yet, will be bonded with ITkPix,the first full-scale \SI{65}{nm} hybrid pixel-readout chip and is developed by the RD53 collaboration.
        For the sensor a valueable option is using 3D pixels, which have already proved themselves in ATLAS, for the insertable B layer (IBL).\red{qualcosa sui 3d usati a ATLAS.}
        These pixel tracking systems will increase the number of pixels of a factor about 7, passing from 92 milioni to 6 miliardi. 

        %3D silicon sensor technology has been chosen to instrument the innermost pixel layer of ITk, which is the most exposed to radiation damage. 50  50 and 25  100 .D
        %sensors are an established technology that has been already
        %employed in experiments at the LHC such as in the ATLAS
        %Insertable B-Layer (IBL)  and for the tracker of the AFP
        %experiment . With respect to these designs the new ITk 3D
        %sensors feature a reduced pixel cell size of 25  100 and 50 
        %50 um 2 with one collecting electrode .  active substrate of these new
        %sensors is reduced to 150 um in comparison to the previous
        %generation of 230 um thick 3D sensor
        \vspace{5mm}
        
        \textbf{CMS}\\

        \vspace{5mm}
        \textbf{LHCb} \\
        LHCb is a dedicated heavy-flavour physics experiment by exploiting pp interactions at \SI{14}{TeV} at LHC. 
        It was the last experiment to upgrade the vertex detector Vertex Locator (VELO) replacing the silicon-strip with pixels in May 2022. 
        As the instantaneous luminosity in Run3 is increased by a factor $\lesssim$10, much of the readout electronics and of the trigger system have been developed in order to cope with the large interaction rate.
        To place the detector as close as possible to the beampipe and reach a better track reconstruction resolution, the VELO can be moved: during the injection of LHC protons it is parket at \SI{3}{cm} from the beams and only when the stability is reach it is brought at $\sim$\SI{5}{mm}. Radiation hardness as well as readout speed are then a priority for the detectors: that's why the collaboration opted for a hybrid system. 
        The Velopix is made bonding sensors, each measuring 55 $\times$ 55 micrometers, \SI{200}{\um}-thick to a \SI{200}{\um}-thick ASIC specially developed for LHCb and coming from the Medipix family (sec. \ref{sec:}).
        \red{It is capable of handling up to 900 million hits per second per chip, while withstanding the intense radiation environment. The new VeloPix readout chips have a readout speeds of up to 20 Gb/s each, resulting in 3 Tb/s torrent of data }
        Since the detector is operated under vacuum near the beam pipe, the heat removal is particularly difficult and evaporative CO2 microchannel cooling are used. 

    \subsection{A DEPFET example: Belle-II}
        

    \subsection{CMOS MAPS: ALICE and STAR}
        \textbf{ALICE}\\
        ALICE (A Large Ion Collider Experiment) is a detector dedicated to heavy-ion physics and to the study of the condensed phase of the chromodynamics at the LHC.
        The tracking detector consists of the Inner Tracking System (ITS), the gaseous Time Projection Chamber (TPC) and the Transition Radiation Detector (TRD) and those are embedded in a magnetic field of \SI{0.5}{T}. The ITS is made by six layers of detectors, two for each type, from the interaction point outwards: Silicon Pixel Detector (SPD), Silicon Drift Detector (SDD) and Silicon Strip Detector (SSD).         
        Contrary to the others LHC experiments, ALICE tracker in placed in a quite different environments: the expected dose is smaller by two order of magnitude and the rate of interactions is few \si{MHz} instead of \SI{40}{MHz}, but the number of particles comes out of each interaction is higher (the SPS is invested by a density of particles of $\sim$\SI{100}{1/cm\squared}).  
        The reconstruction of very complicated events whit a large number of particle is a challenge, hence to segment and to minimize the amount of material, which may cause secondary interaction complicating futher the event topology, is considered a viable strategy. 

        \red{Upgrade con Monopix1}
        Thanks to the reduction of the material budget, ITS2, which uses the ALPIDE chip developed by ALICE collaboration, obtained an amazing improvement both in the position measurement and in the momentum resolution, improving the efficiency pf track reconstruction for particle with very low transverse momentum (by a factor 6 at pT $\sim$ 0.1 GeV/c). Further advancements in CMOS MAPS technology are being aggressively pursued for the ALICE ITS3 vertex detector upgrades (foreseen around 2026-27), with the goals of further reducing the sensor thickness and improving the readout speed of the devices, while keeping power consumption at a minimum.
        \vspace{5mm}
        \textbf{STAR}\\
        MIMOSA-28 devices for the first MAPS-based vertex detector: a 356 Mpixel two-layer barrel system for the STAR experiment at Brookhaven’s Relativistic Heavy Ion Collide

\section{Application in imaging}
    \subsection{Medipix}
    Medipix is a family of read-out chips for particle imaging and detection developed by the Medipix Collaborations. The original concept is that it works like a camera, detecting and counting each individual particle hitting the pixels when its electronic shutter is open. This enables high-resolution, high-contrast, noise hit free images - making it unique for imaging applications
    Medipix2 Collaboration was started in 1999, the Medipix3 collaboration in 2005 and finally the Medipix4 collaboration in 2016. 
    In generale essendo single photon counting ci sono un sacco di applicazioni: fai un elenco. 

    Dalla collaborazione Medipix2 è nato anche il chip Timepix

    Medipix2 250 mm CMOS process\\
    Medipix3 chip in 130nm CMOS.\\
    The aim of the Medipix4 collaboration is designing pixel read-out chips that for the first time are fully prepared for TSV processing and may be tiled on all four sides. 
    Two new chips are foreseen:Medipix4, which will target spectroscopic X-ray imaging at rates compatible with medical CT scans, and Timepix4, which will provide particle identification and tracking with higher spatial and timing precision.

    Utilizzi in medicina: 
    Radiography and computed tomography (CT) use X-ray photons to study the human body. The Medipix chips that implement on-pixel single photon counting provide many advantages for use in these fields. The technology has been applied in X-ray CT, in prototype systems for digital mammography, in CT imagers for mammography and for beta- and gamma-autoradiography of biological samples. , the images are no longer black and white
    %A noteworthy example of detector originally used in particle physics, and later employed mainly for medical imaging, but also in space and for art authentication, is Medipix, a hybrid system developed at CERN within the Medipix collaboration usato anche da LHC-b (Timepix).\\

    Timepix is being exploited for radiation monitoring in NASA's orion rocket and at the International Space Station. 

    As an example, the chips are now being used in the ATLAS experiment to provide independent, real-time information about the radiation environment in the experimental cavern - in principle the same task that Timepix does at the International Space Station. A direct 'spin back' to high-energy physics is VELOpix, based on Timepix3. Benefiting from developments made in the framework of the Medipix3 consortium which were not originally intended for high-energy physics, VELOpix will serve as the read-out chip in the new vertex detector of tje LHCb experiment, which is planned for installation in 2018.
    
    
    \subsection{Applicability to FLASH radiotherapy}