\section{Threshold and noise: figure of merit for pixel detectors}
\red{IN QUESTO CAPITOLO HO MESSO SOLO APPUNTI SPARSI DA RIORGANIZZARE, E DEVO AGGIUNGERE POI I PLOT DI MONOPIX1}

The signal to threshold ratio is the figure of merit for pixel detectors.\\

la soglia deve essere abb alta da tagliare il rumore ma abb bassa da non perdere efficienza.\\
Invece di prendere il rapporto segnale rumore prendi il rapporto segnale soglia. Perchè?
la soglia è collegato al rumore, nel senso che: supponiamo di volere un occupancy di 10-4
allora sceglierò la soglia in base a questo. (plot su quaderno)
Da questo conto trovo la minima soglia mettibile\\
In realtà quello che faccio è mettere una soglia un po' più grande perchè il rate di rumore
dipende da molti fattori quali la temperatura, l anneling ecc, e non voglio che cambiando leggermente
uno di questi parametri vedo alzarsi molto il rate di rumore. In realtà non è solo il
rumore sensibile a diversi fattori, ma anche la soglia: ad esempio la cosa classica è
la variabilità della soglia da pixel a pixel.\\
In questo modo rumore e soglia diventano parenti.\\
Review pag 26.


Questo implica tra le altre cose che voglio poter assegnare delle soglie diverse
a diversi pixel: Drawback è dare spazio per registri e quantaltro.\\
Questo lascia però ancora aperto il problema temporale delle variazioni del rumore:
problema per cui diventano necessarie le misure dei sensori dopo l'irraggiamento.\\


Non fare trimming sulla soglia è uno dei problemi che si sono sempre incontrati: a casusa dei mismatch dei transistor
le soglie efficaci pixel per pixel cambiano tanto.
La larghezza della s curve è il noise se assumi che il noise è gaussiano


Il trimming della soglia avviene con dei DAC: la dispersione della soglia dopo al tuning e dovuta al dac è: 
\begin{equation}
    \sigma_{THR, tuned} = \frac{\sigma_{THR}}{2^{n bit}}
\end{equation}
dove il numero di bit cambia varia tra 3-7 tipicamente. Monopix è 7 Arcadia 6\\

Each ROIC is different in this respect, but in general the minimum stable threshold was around 2500 electrons (e) in 1st generation ROICs, whereas it will be around 500 e for the 3rd generation. This reduction has been deliberate: required by decreasing input signal values. Large pixels (2  104 um2), thick sensors (maggiore di200 um), and moderate sensor radiation damage for 1st generation detectors translated into expected signals of order 10 ke, while small pixels (0.25  104 um2), thinner sensors (100 um), and heavier sensor radiation damage will lead to signals as low as 2 ke atthe HL-LHC\\
The ENC can be directly calculated by the Cumulative Distribution Function (CDF) (scurve) obtained from the discriminator "hit" pulse response to multiple charge injections

Radiation damages oxide layer causes shift of MOSFET threshold voltage
\section{TJ-Monopix1 characterization}

%python3 -i calibration/scurve_tot_histo.py -f calibration/calibration_data/20220506 -i 1 100 questo script crea i file di output contenenti i parametri dei fit del tot e della s curve
%python3 -i calibration/tot_charge_plotting.py -f calibration/calibration_data/20220506 per fare il plot della s curve e relativi residui 
%python3 -i calibration/tot_histo2d.py -f calibration/calibration_data/20220506 per fare l'istogramma 2d del tot

    \subsection{Threshold and noise dispersion}
    Un plot con s curve e residui (perchè dovrebbe essere migliroe il modello con doppia retta? sul RD53 c'era scritto, trovalo e leggilo)
    Istogrammi e colormap\\
    
    \subsection{Absolute calibration of ToT}
    Misure con il ferro. Metti un plot di singolo pixel dello spettro del ferro fittato con CB. Perchè CB? rimuovere i cluster comunque lasciava una coda abbastanza grande a sx e fittare con una gaussiana comunque non dava risultati migliori. 
    
%python3 -i acquisition_Fe55/fit_tot_single_pixel.py -f acquisition_Fe55/source_PMOSS/ per fare il fit    
%python3 -i acquisition_Fe55/plot_tot_single_pixel.py -f acquisition_Fe55/source_PMOSS/ -fl 'gauss_line' per fare il plot di single pixel



\section{ARCADIA-MD1 characterization}
