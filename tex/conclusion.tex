In this thesis I have presented the characterization of two Monolithic Active Pixel Devices, the TJ-Monopix1 and the ARCADIA main demonstrator.
They are prototypes still in an initial development phase and the purpose of this characterization is to help understand the detailed operation of the devices, not to reach clear conclusions about their suitability for specific applications.
For both devices I contributed to the setup of the test environment in the INFN clean laboratories and carried out the measurements personally.

Concerning TJ-Monopix1 the values found are reasonably in agreement with the simulations, although the threshold and noise ($\sim$\SI{400}{\elementarycharge}$^-$ and $\sim$\SI{15}{\elementarycharge}$^-$) have been found to be higher than the expected values ($\sim$\SI{270}{\elementarycharge}$^-$ and $\sim$\SI{9}{\elementarycharge}$^-$).
This difference is not too surprising,  and can be justified considering that the simulations were performed with the front end in a optimized status, while in our measurements the front end working point optimization was limited by the need to keep under control the number of noisy pixels.
%%\red{dovuta anche al fatto di non poter scendere a soglie troppo basse per via di un numero troppo alto di noisy pixel che ne disabilitava poi troppi.}

The threshold dispersion was measured to be $\sim$\SI{30}{\elementarycharge}$^-$, in agreement with the simulation; the dispersion across the matrix can be reduced and can be make comparable with the ENC by adding a bit for trimming on each pixel.
%\red{(equazione 4.3)}.
TJ-Monopix1 was tested with Fe55 and Sr90 sources, and with cosmic rays, allowing the absolute calibration of the injection circuit and a first characterization of the chip response to radiation.
An initial test of the device response to a high rate FLASH beam was also performed, although the limitations of the chip prototype prevented reaching conclusions on the suitability of the device for this application.


Regarding the ARCADIA-MD1 prototype the very preliminary results have shown that its behavior is in agreement with what expected and, after a complete characterization of the front end, the test of its pioneering readout mode and of its coarse hardware clustering algorithm depending on the operating range will be certainly a main target.

The R$\&$D of monolithic active devices is and important and active sector since they represent a low-cost and versatile technology, with possible future applications in many field and, as stated several times, they will possible open new scenarios particle detectors both in accelerator experiments and in medical physics.
The future perspective is now the development of bigger and faster devices for what concerns the HEP experiments, and a more detailed study of the sensor itself, studying the fabrication parameters and focusing on the operating limits at high dose rate, for what concerns the application in dosimetry.