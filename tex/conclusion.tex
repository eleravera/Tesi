Since MAPS are devices that need fully to be tested and whose development is ongoing, a defined and specific application is not the target of this thesys, but I certainly conclude commenting on the results of the measurements done up to now. 
 
Concerning TJ-Monopix1 the values found are reasonably in agreement with the simulations, despite the noise and the threshold ($\sim$\SI{400}{\elementarycharge}$^-$ and $\sim$\SI{15}{\elementarycharge}$^-$) have been found to be higher than the expected values ($\sim$\SI{270}{\elementarycharge}$^-$ and $\sim$\SI{9}{\elementarycharge}$^-$). In fact, this difference is reasonable and can be justify considering that the simulations were performed with the front end in a optimize status, while in our measurement there has not been a complete optimization. \red{dovuta anche al fatto di non poter scendere a soglie troppo basse per via di un numero troppo alto di noisy pixel che ne disabilitava poi troppi.}

The threshold dispersion was measured to be $\sim$\SI{30}{\elementarycharge}$^-$, in agreement with the simulation; the dispersion across the matrix can be reduced and can be make comparable with the ENC by adding a bit for trimming on each pixel \red{(equazione 4.3)}. Moreover this would allow injecting a charge up to 128$\times$2$\times$\SI{10}{\elementarycharge}$^-$= \SI{2560}{\elementarycharge}$^-$ (each DAC would corresponds to \SI{10}{\elementarycharge}$^-$) and would be beneficial also as regards the characterization of the device helping in the interpretation of the spectrum of Sr90 and cosmic rays, for that the peak position is not yet understood.
Regarding the ARCADIA-MD1 prototype the very preliminary results have shown that its behavior is in agreement with what expected and, after a complete characterization of the front end, the test of its pioneering readout mode and of its coarse hardware clustering algorithm depending on the operating range will be certainly a main target.

The R$\&$D of monolithic active devices is and important and active sector since they represent a low-cost and a versatile technology, with possible future applications in many field and, as stated several times, they will possible open new scenarios particle detectors both in accelerator experiments and in medical physics.
The future perspective is now the development of bigger and faster devices for what concerns the HEP experiments, and a more detailed study of the sensor itself, studying the fabrication parameters and focusing on the operating limits at high dose rate, for what concerns the application in dosimetry.

