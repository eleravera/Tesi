Since the 1980s, when the fabrication of device with very small electrodes (50-100 \si{\um}) became a practical possibility, pixel detectors have been widely employed for imaging and tracking charged particles in the vertex region of experiments at accelerators. Thanks to their excellent spatial resolution, today even better than \SI{10}{\um}, they allow for true three dimensional space-point determination even at high particle fluxes and in particular for the identification of secondary vertices of short-lived particles such as $\tau$ and B mesons.
Requirement imposed by accelerators are stringent and they will become even more so with the increase of luminosity; in this scenario CMOS Monolithic Active Pixel Sensors (MAPS), based on the technology of CMOS cameras, are being developed to improve the performance of the hybrid pixel detectors, which currently constitute the state-of-art for large scale pixel detector, in particular by reducing the amount of material, power consumption and pixel dimension. Indeed, while hybrid pixels are made by two parts, the sensor and the electronics, welded together through microconnections, the MAPS integrate them all on the same wafer.

Experiments such as ALICE at LHC and STAR at RHIC have already introduced the CMOS MAPS technology in their detectors. ALICE Tracking System (ITS2), upgraded during the LHC long shut down in 2019-20, was the first large-area ($\sim$10 \si{m^2}) silicon vertex detector based on CMOS MAPS. Thanks to the reduction of the material budget, ITS2, which uses the ALPIDE chip developed by ALICE collaboration, obtained an amazing improvement both in the position measurement and in the momentum resolution, improving the efficiency of track reconstruction for particle with very low transverse momentum (by a factor 6 at $p_{T}\sim$ 0.1 \si{GeV/c}).
Further advancements in CMOS MAPS technology are being aggressively pursued for the ALICE ITS3 and the Belle II vertex detector upgrades (both foreseen around 2026-27), and by the R$\&$D53 collaboration for the upgrade at HL-LHC, with the goals of further reducing the sensor thickness and improving the readout speed of the devices, while keeping power consumption at a minimum.

Beside tracking, the development of pixel detectors is a very active field with many applications: a noteworthy example of detector originally used in particle physics and later employed for medical imaging, in space detectors and for art authentication, is Medipix, a hybrid system developed at CERN within the Medipix collaboration.
Among medical applications, a possible use of CMOS MAPS could be in dosimetry: in the last few years the search of radiotherapy oncological treatments with high intensity beams (FLASH mode) is requiring new dosimeters, both for the therapies as well as new beam-monitors (especially for focused very high energy electron beams), which are capable of deal with extreme dose rate (up to 40 \si{Gy/s}).

I have studied the characteristics of two ALPIDE-like CMOS MAPS chips and tested them under different front end configuration. The first chip, the TJ-Monopix1 from the Monopix series, is a TowerJazz MAPS fabricated in 180 nm CMOS technology with an active area of 1$\times$2~\si{cm\squared} (448$\times$224 pixels) and is one of the prototypes for the Belle II vertex detector upgrade. The second chip, called Main Demonstrator-1, has an active area of 1.28$\times$1.28~\si{cm\squared} (512$\times$512 pixels) is produced by LFoundry in 110 nm CMOS technology and designed by the ARCADIA (Advanced Readout CMOS Architectures with Depleted Integrated sensor Arrays) group; it is intended to be a general purpose device with possible use in medical scanners, space experiments, future lepton colliders and also possibly X-ray applications with thick substrates.  
The main differences between the two chips are in the output signal type and in the readout sequence of the matrix. Concerning the former, TJ-Monopix1 returns an analog output information, that is the time over threshold of the pulse, which can be related with the charge released by the particle in the sensor, while MD1 returns only a digital information; regarding the latter, instead, TJ-Monopix1 has a completely sequential readout, while MD1 roughly combines the information of the hits before the readout in order to reduce the data transmission time.

I have set up the test systems for the two chips in the INFN clean laboratories and characterized the devices electrically and with radioactive sources in terms of threshold, noise, dead time and analog response.
The mean minimum stable threshold evolved through different generation of chips and nowadays it is less than \SI{500}{e-}, allowing thinner sensors with smaller signals: TJ-Monopix1 has proven to be in agreement with this trend, having a threshold of $\sim$\SI{400}{e-}, to be compared with the \SI{2000}{e-} signal expected for a minimum ionizing particle in an epitaxial layer of \SI{25}{\um}. 
Moreover, since one of the main challenges of MAPS are the differences between pixels due to process parameters variation across the wafer, which make the sensor response nonuniform, I have measured the threshold and noise dispersion across the matrix, which I found to be \SI{40}{e-} and \SI{2}{e-} respectively.
I have also studied the response of the analog signal recorded by TJ-Monopix1, that is the time over threshold, and performed a calibration of its absolute value using a Fe55 X-ray source.
All these measurements are important to verify the design parameters of the chip and to validate the chip simulation. 

As conclusion of the measurement campaign, we have tested TJ-Monopix1 at very high intensity using the electron beam of the new ElectronFlash accelerator designed for both medical research and R$\&$D in FLASH-radiotherapy and recently installed at Santa Chiara hospital in Pisa. I have participated in the design of the setup needed for testbeam measurement and I am currently working on the analysis of the data collected. 