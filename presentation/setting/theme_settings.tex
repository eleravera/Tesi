%%%%%%%%%%%%%%%%%%%%%%%%%%%%%%%%%%%%%%%%%%%%%%%%%%%%%%%%%%%
%
% Copyright 2022 by Isac Pasianotto
%
% This file may be distributed and/or modified
%
% 1. under the LaTeX Project Public License and/or
% 2. under the GNU Public License.
%
%%%%%%%%%%%%%%%%%%%%%%%%%%%%%%%%%%%%%%%%%%%%%%%%%%%%%%%%%%%

%% 	Variabili di tipo "color": 

\definecolor{bluUnits100}{rgb}{0.16,0.22,0.36} 
\definecolor{bluUnits80}{rgb}{0.22,0.3,0.51}
\definecolor{bluUnits70}{rgb}{0.25,0.35,0.58}
\definecolor{bluUnits50}{rgb}{0.41,0.51,0.74}
\definecolor{bluUnits40}{rgb}{0.56,0.63,0.81}
\definecolor{bluUnits25}{rgb}{0.78,0.82,0.9}
\definecolor{bluUnits10}{rgb}{0.93,0.94,0.96}
\definecolor{grey}{rgb}{0.3686, 0.5255, 0.6235} 

%%	Palette di colori 

\setbeamercolor{palette primary}{bg=bluUnits100,fg=white}
\setbeamercolor{palette secondary}{bg=bluUnits80,fg=bluUnits25}
\setbeamercolor{palette tertiary}{bg=bluUnits50,fg=bluUnits100}
\setbeamercolor{palette quaternary}{bg=bluUnits40,fg=white}
\setbeamercolor{palette quaternary blue}{bg=bluUnits40,fg=bluUnits100}
\setbeamercolor{palette light primary}{bg=bluUnits25,fg=bluUnits100}
\setbeamercolor{palette titleframe}{bg=bluUnits10, fg=bluUnits80}


		%%%%%%%%%%%%%%%%%%%%%%%%%%%%%%%%%%
		%% Impostazioni generali slide  %%
		%%%%%%%%%%%%%%%%%%%%%%%%%%%%%%%%%%

%%	Setta l'immagine da mettere come sfondo, riducendone l'opacità
%\usebackgroundtemplate{\tikz\node[opacity=0.1]{\includegraphics[height=\textheight]{\LogoFiligrana}};}

%%	Elenchi puntati, numerati, etc.

\setbeamercolor{structure}{fg=bluUnits80}
\setbeamertemplate{enumerate item}[circle]
\setbeamertemplate{itemize subitem}[ball]
% Valutare a secoda del contesto se sostituire con 
%\setbeamertemplate{items}[circle]
\setbeamertemplate{items_circ}[circle]
\setbeamercolor{alerted text}{fg=bluUnits50}

%% 	Colore delle scritte nella presentazione
\setbeamercolor{normal text}{fg=bluUnits100,bg=white}

%% 	Settaggio della linea in alto (headline)
\setbeamertemplate{headline}{
	\vskip1pt
	\leavevmode	
	\hbox{
		\begin{beamercolorbox}[wd=.99\paperwidth,ht=2.5ex,dp=1.125ex]{palette light primary}
			\insertsectionnavigationhorizontal{\paperwidth}{}{\hskip0pt plus1filll}
		\end{beamercolorbox}
	}
}

%%	Settaggio riga in basso (footline) 
 
\setbeamertemplate{footline}{
       \leavevmode
	   \hbox{
           \begin{beamercolorbox}[wd=.2\textwidth,ht=2.6ex,dp=1ex,center]{palette tertiary}
		    \usebeamerfont{author in head/foot}\insertshortauthor
    	\end{beamercolorbox}
	
    	\begin{beamercolorbox}[wd=.27\textwidth,ht=2.6ex,dp=1ex,center]{palette quaternary}
	   		\usebeamerfont{institute in head/foot}\insertshortinstitute
	   	\end{beamercolorbox}
	
    	\begin{beamercolorbox}[wd=.40\textwidth,ht=2.6ex,dp=1ex,center]{palette primary}
    		\usebeamerfont{title in head/foot}\insertshorttitle
    	\end{beamercolorbox}

    	\begin{beamercolorbox}[wd=.1\textwidth,ht=2.6ex,dp=1ex,center]{palette light primary}
	   		\insertframenumber{}/\inserttotalframenumber
	    \end{beamercolorbox}
    }
    \vskip2pt

}

%%	Settaggio tittoli delle slide  

\setbeamertemplate{frametitle}{
	\begin{beamercolorbox}[wd=\paperwidth,ht=2.75ex,dp=1ex,left]{palette titleframe}
		\qquad \textbf{\insertframetitle}
	\end{beamercolorbox}
}


		%%%%%%%%%%%%%%%%%%%%%%%%%%%%%%%
		%% Impostazioni Prima Slide  %%
		%%%%%%%%%%%%%%%%%%%%%%%%%%%%%%%
	
	
	
	
		
\def\setTitlestyleDissertation{
	
	\defbeamertemplate*{title page}{customized}[1][]{
		
		%  Commentare il seguente ambiente {center} e decommentare {flushright} quello successivo in caso
		%	si voglia usare solo il logo dell'UNI
		\begin{center}
			\medskip
			\includegraphics[width=0.25\textwidth]{\LogoUniversita}
		\end{center}
	
		%	\begin{flushright}
		%		\includegraphics[width=0.45\textwidth]{\LogoUniversita}	
		%	\end{flushright}
	
		\smallskip
		
		\begin{center}		
			\usebeamerfont{title}\textbf{\inserttitle}\par
			\usebeamerfont{subtitle}\usebeamercolor[fg]{subtitle}\insertsubtitle\par
			\medskip		
			
			%% Il seguente layout dentro l'ambiente multicols serve per le tesi.
			
			\begin{multicols}{2}
				\begin{tabular}{c}
					\usebeamerfont{normal text}{\relatoreLabel} \\
					\usebeamerfont{author}{\relatore}\\
					\usebeamerfont{author}{\correlatore}

						% Decommentare in caso siano presenti dei correlatori
					%\\
					%\usebeamerfont{normal text}{\correlatoreLabel} \\
					%\usebeamerfont{author}{\correlatore}
						
				\end{tabular}					
				\columnbreak
				\begin{tabular}{c}
					\candidatoLabel \\
					\usebeamerfont{author}{\insertauthor}
				\end{tabular}
			\end{multicols}
		
			\par
			
			\bigskip  	% --> nel caso di relatore e basta
			%\smallskip 	% --> nel caso di relatore + correlatore
			
			\insertinstitute\par
			
			\bigskip	% --> nel caso di relatore e basta
			%\smallskip	% --> nel caso di relatore + correlatore
			
			\usebeamerfont{date}\insertdate\par
			
			\bigskip	% --> nel caso di relatore e basta
			%\smallskip	% --> nel caso di relatore + correlatore
		\end{center}
	}
}


