\setbeamertemplate{headline}[backup]
\setbeamertemplate{footline}[backup]





    %%%%%%%%%%%%%%%%%%%%%%%%%%%%%%%%%%%%%%%%
    %%  Slide 1: <BACKUP>  %%
    %%%%%%%%%%%%%%%%%%%%%%%%%%%%%%%%%%%%%%%%
    \begin{frame}[noframenumbering]
        Backup        
    \end{frame} 


    %%%%%%%%%%%%%%%%%%%%%%%%%%%%%%%%%%%%%%%%
    %%  Slide 1: <Applications>  %%
    %%%%%%%%%%%%%%%%%%%%%%%%%%%%%%%%%%%%%%%%
    %\begin{frame}
    %    \frametitle{Tracking in HEP}

    %    Task of pixel detectors in tracking:
    %    \begin{itemize}
    %        \item pattern recognition with the identification of particle tracks even in the presence of large backgrounds and pile-up
    %        \item measurement of vertices (primary and secondary)
    %        \item multi-track and vertex separation in the core of jets
    %        \item measurement of specific ionization
    %        \item momentum measurement combining with the information from other detectors
    %    \end{itemize}
    %\end{frame} 


    %%%%%%%%%%%%%%%%%%%%%%%%%%%%%%%%%%%%%%%%
    %%  Slide 1: <>  %%
    %%%%%%%%%%%%%%%%%%%%%%%%%%%%%%%%%%%%%%%%
    \begin{frame}[noframenumbering]
        \frametitle{}
        \begin{figure}[h!]
            \centering
            \includegraphics[width=.8\linewidth]{figures/Monopix1/readout_timing.png}
        \end{figure}
    \end{frame} 

    %%%%%%%%%%%%%%%%%%%%%%%%%%%%%%%%%%%%%%%%
    %%  Slide 1: <>  %%
    %%%%%%%%%%%%%%%%%%%%%%%%%%%%%%%%%%%%%%%%
    \begin{frame}[noframenumbering]
        \frametitle{Radiotherapy}
        \begin{figure}[h!]
            \centering
            \includegraphics[width=.8\linewidth]{figures/pixel_detectors_usage/Bragg-Peak.png}
        \end{figure}
    \end{frame} 


     %%%%%%%%%%%%%%%%%%%%%%%%%%%%%%%%%%%%%%%%
    %%  Slide 1: <>  %%
    %%%%%%%%%%%%%%%%%%%%%%%%%%%%%%%%%%%%%%%%
    \begin{frame}[noframenumbering]
        \frametitle{Front end}
            \textbf{ALPIDE like}
            \begin{figure}[h!]
                \centering
                \includegraphics[width=.8\linewidth]{figures/Monopix1/Monopix1_FE_circuit.png}        
            \end{figure}
    \end{frame}     

    
    %%%%%%%%%%%%%%%%%%%%%%%%%%%%%%%%%%%%%%%%
    %%  Slide 3: <Sensors types>  %%
    %%%%%%%%%%%%%%%%%%%%%%%%%%%%%%%%%%%%%%%%
    \begin{frame}[noframenumbering]
        \frametitle{MAPS sensor types}
            \begin{itemize}
                \item \textbf{Large fill factor} ($\sim$100-200 \si{fF}) or \textbf{small fill factor} ($<$\SI{5}{fF}), depending on the deep p-well structures
            \end{itemize}
            %\centering
            \includegraphics[width=1.05\linewidth]{figures/Pixel_detectors/large_small_sensor_scheme.png}\\
            \begin{columns}
                \column{0.5\textwidth}  
                    %\centering
                    \includegraphics[width=1.1\linewidth]{figures/Pixel_detectors/ALPIDE_after_PM.png}
                \column{0.5\textwidth}  
                    \begin{itemize}
                        \item \textbf{Process modification} with a low dose planar implant. \\
                        whose main investigator is ALICE\\
                        %Advantage: no need of change in the sensor and circuit layout
                    \end{itemize} 
            \end{columns}

            \end{frame} 

    %%%%%%%%%%%%%%%%%%%%%%%%%%%%%%%%%%%%%%%%
    %%  Slide 4: <ToT calibration>  %%
    %%%%%%%%%%%%%%%%%%%%%%%%%%%%%%%%%%%%%%%%
    \begin{frame}[noframenumbering]
        \frametitle{Different collection properties}
        2 different \textbf{dose profile} in the sensor: Reduced Deep P-Well (RDPW) and Full Deep P-Well (FDPW)

        \begin{itemize}
            \item Different charge collection efficiency due to the deep p-well structure within the sensor
        \end{itemize}
        \begin{columns}
            \column{0.5\textwidth}                  
                \includegraphics[width=1.1\linewidth]{figures/charaterization/fit_gauss_r185.pdf}
                \centering Full deep p-well
            \column{0.5\textwidth}    
                \includegraphics[width=1.1\linewidth]{figures/charaterization/fit_line_gauss_r69.pdf}
                \centering Partial deep p-well
        \end{columns}
    \end{frame}

    %%%%%%%%%%%%%%%%%%%%%%%%%%%%%%%%%%%%%%%%
    %%  Slide 4: <Aquisitions with sources>  %%
    %%%%%%%%%%%%%%%%%%%%%%%%%%%%%%%%%%%%%%%%
    \begin{frame}[noframenumbering]
        \frametitle{Acquisition with sources: Fe$^{55}$, Sr$^{90}$, cosmic rays}
        \begin{columns}
            \column{0.35\textwidth} 
                \begin{table}
                    \tiny
                    \begin{center}
                    \begin{tabular}{|c | c |}
                    \hline
                    Source & Rate [\si{Hz}]\\
                    \hline
                    \hline
                    Fe$^{55}$ & 3.3 10$^3$\\
                    Sr$^{90}$ & \\
                    CR & \\
                    Noise &  \\
                    \hline
                    \end{tabular}
                    \end{center}
                \end{table}

            \column{0.65\textwidth} 
                \begin{itemize}
                    \item Fe$^{55}$ photon produces smaller clusters: pure charge sharing
                    \item Sr$^{90}$ and cosmic rays go across more pixels 
                \end{itemize}    
        \end{columns}
        \medskip
        \begin{columns}
            \column{0.5\textwidth} 
            \includegraphics[width=1.\linewidth]{figures/charaterization/hits_per_evt.png}            
            \column{0.5\textwidth} 
            \includegraphics[width=1.\linewidth]{figures/charaterization/cluster_dimension.pdf}
        \end{columns}
    \end{frame}    

    %%%%%%%%%%%%%%%%%%%%%%%%%%%%%%%%%%%%%%%%
    %%  Slide 1: <>  %%
    %%%%%%%%%%%%%%%%%%%%%%%%%%%%%%%%%%%%%%%%
    \begin{frame}[noframenumbering]
        \frametitle{Noisy pixels}
        \begin{columns}
            \column{0.4\textwidth} 
                \centering
                \includegraphics[width=.8\linewidth]{figures/Monopix1/masking_scheme.png}        
            \column{0.4\textwidth} 
            %\includegraphics[width=.8\linewidth]{figures/characterization/noisy.png}
        \end{columns}
    \end{frame}    


    %%%%%%%%%%%%%%%%%%%%%%%%%%%%%%%%%%%%%%%%
    %%  Slide 1: <MIMOSA>  %%
    %%%%%%%%%%%%%%%%%%%%%%%%%%%%%%%%%%%%%%%%
    \begin{frame}[noframenumbering]
        \frametitle{MIMOSA series}
        First CMOS MAPS in HEP. Ma aveva rolling shut readout. 
        \includegraphics[width=.45\linewidth]{figures/pixel_detectors_usage/ALICE_FoCAL.png} with MIMOSA
    \end{frame} 


    %%%%%%%%%%%%%%%%%%%%%%%%%%%%%%%%%%%%%%%%
    %%  Slide 1: <ALICE>  %%
    %%%%%%%%%%%%%%%%%%%%%%%%%%%%%%%%%%%%%%%%
    \begin{frame}[noframenumbering]
        \frametitle{ALPIDE - ALice PIxel DEtector}
        ALICE ITS2 upgraded in 2019-20\\
        \smallskip
        The \textbf{sensor} uses high resistivity p-type epi-layer, TowerJazz in \SI{0.18}{\um}. It is the first large area $\sim$\SI{10}{m\squared} MAPS detector with sparsified readout.
        Many MAPS have an \textbf{ALPIDE-based Front End} (i.e. TJ-Monopix1, ARCADIA)
        \begin{columns}
            \column{0.35\textwidth}  
                \includegraphics[width=1.3\linewidth]{figures/pixel_detectors_usage/alice.png}
            \column{0.6\textwidth}
            \begin{itemize}
                \item position resolution $\sim$\SI{5}{\um} (pixel dimension 27$\times$29\si{\um\squared})
                \item X$_0$/layer reduced from 1.14\% to 0.3\%
                \item tracking efficiency of low-p$_T$ ($p_T \sim$\SI{0.1}{GeV/c}) improved by a factor 6
            \end{itemize}
        \end{columns}
        %ALPIDE is under test for several other HEP detectors (i.e. Belle2) and applications and 
    \end{frame} 


    %%%%%%%%%%%%%%%%%%%%%%%%%%%%%%%%%%%%
    %% Slide 3: <> %%
    %%%%%%%%%%%%%%%%%%%%%%%%%%%%%%%%%%%%
    \begin{frame}[noframenumbering]
        \frametitle{Test on beam: preliminary results}
        \begin{itemize}
            \item With \textbf{both} the collimators, DPP=\SI{0.07}{Gy}, t$_p$=\SI{4}{\us}, PRF=\SI{1}{Hz}
        \end{itemize}
        \medskip
        \begin{figure}
            \includegraphics[width=0.49\linewidth]{figures/test_beam/Q1_17_11.pdf}
            \includegraphics[width=.49\linewidth]{figures/test_beam/Q2_17_11.pdf}
        \end{figure}
        \begin{itemize}
            \item the collimators do not shield the detectors by all particles
            \item probably photons are produced by electrons in Al collimators
            \item for each accelerator pulse, 2 readout "cycle"  
        \end{itemize}
    \end{frame} 
    
    %%%%%%%%%%%%%%%%%%%%%%%%%%%%%%%%%%%%
    % Slide 3: <> %%
    %%%%%%%%%%%%%%%%%%%%%%%%%%%%%%%%%%%%
    \begin{frame}[noframenumbering]
        \frametitle{Test on beam: preliminary results}
        \begin{itemize}
            \item\textbf{Without} any collimator, DPP=\SI{0.04}{Gy}, t$_p$=\SI{4}{\us}, PRF=\SI{1}{Hz}
            \item MIP are expected to release \SI{2000}{\elementarycharge}$^-$, and because of rollorver are expected to be 300-400\si{\elementarycharge}$^-$
        \end{itemize}
        \medskip
        \begin{columns}
            \column{0.5\textwidth} 
                \includegraphics[width=1.1\linewidth]{figures/test_beam/Qe_17_32.pdf}
            \column{0.5\textwidth} 
                \begin{itemize}
                    \item ToT converted in charge
                    \item pixels turn on in N clock counts
                    \item after each pulse an induced signal on the whole matrix
                \end{itemize}
        \end{columns}   
        \medskip
        Need for a simulation to understand the data
    \end{frame}   


    %%%%%%%%%%%%%%%%%%%%%%%%%%%%%%%%%%%%
    %% Slide 2: <> %%
    %%%%%%%%%%%%%%%%%%%%%%%%%%%%%%%%%%%%
    \begin{frame}[noframenumbering]
        \frametitle{Test on the beam: ElectronFLASH}
        ElectronFLASH is the new accelerator for research on FLASH-RT placed in S. Chiara hospital in Pisa \\
        \medskip
        Accelerator charateristics: 
        \begin{itemize}
            \item linear accelerator
            \item bunched beam
            \item two energy configurations 7-9\si{MeV}
            \item can reach ultra high intensity (over \SI{5000}{Gy/s})
            \item beam parameters can be configured independently from each other
            \item equipped with a set of plexiglass applicators (diameters in range from \SI{1}{cm} to \SI{12}{cm}) which are used to produce a uniform dose profile 
        \end{itemize}
    \end{frame}  