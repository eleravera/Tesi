\section{Pixel detectors}



    %%%%%%%%%%%%%%%%%%%%%%%%%%%%%%%%%%%%%%%%
    %%  Slide 1: <Pixels types>  %%
    %%%%%%%%%%%%%%%%%%%%%%%%%%%%%%%%%%%%%%%%
    \begin{frame}
        \frametitle{Pixel detectors: many different types}
        Originally born for \textbf{imaging}, then used for \textbf{tracking} replaced gas detectors because of higher spatial resolution to reconstruct short-lived particle decays
        \begin{columns}
            \column{0.5\textwidth} 
                \vspace*{-0.5cm}\begin{itemize}
                    \item CCDs 
                    \item Hybrid pixels 
                    \item Monolitich Active Pixels
                    \begin{itemize}
                        \item DEPFET %$\rightarrow$ BelleII vertex , perchè per Belle2 è importante il material budget
                        \item CMOS MAPS
                    \end{itemize}
                \end{itemize}
            \medskip
            \pause
            \column{0.5\textwidth} 
                \vspace*{+0.5cm}\centering\includegraphics[width=.9\linewidth]{figures/Pixel_detectors/CCD_presentation.pdf} 
        \end{columns}
        
        \pause
        \vspace*{-0.8cm}\includegraphics[width=.34\linewidth]{figures/Pixel_detectors/hybrid_scheme2.png}
        \pause
        \includegraphics[width=.64\linewidth]{figures/Pixel_detectors/Monolitich.pdf}
    \end{frame}


    %%%%%%%%%%%%%%%%%%%%%%%%%%%%%%%%%%%%%%%%
    %%  Slide 1: <Pixels types>  %%
    %%%%%%%%%%%%%%%%%%%%%%%%%%%%%%%%%%%%%%%%
    \begin{frame}
        \frametitle{Hybrid vs monolithic}
        \begin{columns}
            \column{0.47\textwidth} 
                \centering\textbf{Hybrid pixels}
                \begin{beamercolorbox}[ rounded=true, center]{celadon}
                    \begin{enumerate}
                        \item sensor and ASIC can be optimized separately
                        \item module area$\sim$\SI{10}{cm\squared}
                        \item rate capability$\sim$\si{GHz/cm\squared}
                        \item are more radiation hard 
                    \end{enumerate} 
                \end{beamercolorbox}
            \column{0.06\textwidth} 
                \includegraphics[width=1.5\linewidth]{presentation/pollicesu.png}
            \column{0.47\textwidth} 
                \centering\textbf{Monolithic active pixels}
                \begin{beamercolorbox}[ rounded=true, center]{celadon}
                    \begin{enumerate}
                        \item active thick$\sim$\SI{25}{\um}
                        \item pitch$\sim$10-\SI{50}{\um}
                        \item point resolution$\sim$5-\SI{10}{\um}
                        \item low power consumption
                        \item good S/N ratio
                    \end{enumerate}    
                \end{beamercolorbox}
        \end{columns}
       
        \begin{columns}                
            \column{0.47\textwidth} 
                %\medskip
                \begin{beamercolorbox}[rounded=true, center]{gray}
                    \begin{enumerate}
                        \item pitch$\sim$50-\SI{100}{\um}
                        \item point resolution$\sim$10-\SI{15}{\um}
                        \item \tikzmark{start}sensor thick$\sim$\SI{250}{\um}\tikzmark{end}\tikz[remember picture] \node[coordinate] (t1) {};
                        %\item sensor \tikzmark{start}thick$\sim$\tikzmark{end}\SI{250}{\um}\tikz[remember picture] \node[coordinate] (t1) {};
                        \item bump bonding delicate and expensive
                    \end{enumerate}    
                \end{beamercolorbox}
                %\vspace*{-0.3cm}
                \hspace*{+1.0cm}
                {\color{red}Multiple Scattering!} \tikz[remember picture] \node[coordinate] (n1) {};

                \begin{tikzpicture}[remember picture,overlay]
                    \node<2>[ellipse,draw,fit={(pic cs:start) (pic cs:end)}, red,xscale=.9,yscale=2.] {};
                \end{tikzpicture} 
                
                %\begin{tikzpicture}[remember picture,overlay]
                %    \node<2>[ellipse,draw,fit={(pic cs:start) (pic cs:end)}, red,xscale=2.,yscale=2.] {};
                %\end{tikzpicture} 

                    %disegno la freccia
                    \begin{tikzpicture}[remember picture, overlay]
                        \usetikzlibrary {arrows.meta}
                        \path[draw=red,thick,->]<1-> ([yshift=1mm, xshift=5mm]t1) to [out=0, in=0,distance=0.5in] (n1);
                    \end{tikzpicture}
            \column{0.06\textwidth} 
                \includegraphics[width=1.5\linewidth]{presentation/pollicegiu.png}
            \column{0.47\textwidth}
                \begin{beamercolorbox}[rounded=true, center]{gray}
                    \begin{enumerate}
                    \item module area$\sim$\si{cm\squared}
                    \item rate capability $\sim$\SI{100}{MHz/cm\squared}
                    \end{enumerate}    
                \end{beamercolorbox}
        \end{columns}
    \end{frame}
