\documentclass[a4paper]{report}
\usepackage[a4paper,width=150mm,top=25mm,bottom=25mm]{geometry}
\usepackage{siunitx}
\usepackage{lineno} % To obtain line numbers
\usepackage[dvipsnames]{xcolor}

\newcommand{\red}[1]{\textcolor{red}{#1}}

\begin{document}
\linenumbers

Since the 1980s, when the fabrication of very small electrodes (50-100 \si{\mu m}) became a possibility, pixel detectors have been widely employed for the tracking of particles in the vertex region of experiments at accelerators: thanks to their excellent spatial resolution, today even better than 10 \si{\mu m}, they allow the identification of secondary vertices of short-lived particles such as $\tau$ and B mesons. 
CMOS Monolithic Active Pixel Sensors (MAPS), based on the technology of CMOS cameras, have been developed to improve the performance of the widely used hybrid pixel detectors, in particular by reducing the amount of material, power consumption and pixel dimension.  

Two experiments at accelerator that have already introduced the CMOS MAPS technology in their detectors are ALICE at LHC and STAR at RHIC; ALICE Tracking System (ITS2), upgraded during the LHC long shut down in 2019/2020, was the first large-area ($\sim$10 \si{m^2}) silicon vertex detector based on CMOS MAPS. Thanks to the reduction of the material budget, ITS2 obtained an amazing improvement both in the point position measurement and in the momentum resolution, allowing high efficient track recontruction of particle with very low transverse momentum (efficiency for $p_{T}\sim$ 0.1 \si{GeV/c} improved by a factor $\sim$ 6).
For years ALICE have been pioneering MAPS detectors and its sensors are currently state-of-the art in this sector, to the point that most of today's CMOS MAPS chips implement the same FE of ALICE Pixel Detector, and in fact they are commonly called "ALPIDE-like" sensors.
\red{The main objectives for ITS3 (expected for the long shut down starting in 2026), are to futher thin down sensor thickness and to increase the readout capabilities: non stupisce il fatto completely match requirements for future CMOS MAPS detectors e che stiano aprendo le porte a nuove tecnologie.}

Beside tracking, the development of pixel detectors is a very active field with many applications: an example of detector originally used in particle physics detectors, and later employed for medical imaging, but also in space and for art authentication, is Medipix, a hybrid system developed at CERN within the Medipix collaboration.
\red{UNa piccola frase su medipix?}
Among medical applications, a possible use of CMOS MAPS could in dosimetry: in the last few years the search of radiotherapy oncological treatments with high intensity beams (FLASH mode) is requiring new dosimeters, both for the treatment and for radiological protection, as well as new beam-monitors (especially for focused very high energy electron beams), which are capable of deal with extreme dose rate (up to 40 \si{Gy/s}).

I've studied the characteristics of two ALPIDE-like CMOS MAPS chips and tested them under different front end configuration. The first chip, a TJ-Monopix1 from the Monopix series, is a TowerJazz MAPS with 180 nm CMOS technology and is one of the prototypes for Belle II vertex detector upgrade expected in 2026. The second is projected and designed by ARCADIA (Advanced Readout CMOS Architectures with Depleted Integrated sensor Arrays) mainly  for space and medical applications and it is one of the Main Demonstrator-1 produced by LFoundry with 110 nm CMOS technology. 
Since one of the main challenges of MAPS are the differences between pixels, due to process parameters variation across the wafer, which make the sensor response nonuniform, I've performed the characterization of the chips to find the threshold and noise of the sensors and their dispersion among pixels. To make test and characterization I mainly have worked using pixels setting and testing feature available from the DAQ interface, as for example the injection circuit which allows injecting pixels with a known charge. Then, to have an absolute value in electrons of the signal recorded by TJ-Monopix1, I've also performed an absolute calibration using a Fe55 x-ray source. 
Moreover, I've also test TJ-Monopix1 at high dose rate with the FLASH-accelerator recently installed at Santa Chiara hospital in Pisa, and I have participated in the design of the setup needed for test beam measurement.
I've carried out the work in collaboration with both the Belle II and the physical medical groups in Pisa.
\end{document}








