\documentclass[a4paper]{report}
\usepackage[a4paper,width=150mm,top=25mm,bottom=25mm]{geometry}
\usepackage{siunitx}
\usepackage{lineno} % To obtain line numbers


\begin{document}
\linenumbers

Since the 1980s, when the fabrication of very small electrodes (50-100 \unit{\mu m}) became a possibility, pixel detectors have been widely employed for the tracking of particles in the vertex region of experiments at accelerators: thanks to their excellent spatial resolution, today even better than 10 \unit{\mu m}, they allow the identification of secondary vertices of short-lived particles such as $\tau$ and B mesons. 
CMOS MAPS, based on the technology of CMOS cameras, have been developed to improve the performance of the widely used hybrid pixel detectors, in particular by reducing the amount of material, power consumption, pixel dimension.  CURRENT USE OF CMOS MAPS: ALICE; STAR, speed limitation.\\

Future detectors will require thin and fast cmos maps. 
Very active field with many development

CMOS monolithic active pixels (CMOS MAPS), based on the technology of CMOS cameras, are opening promising perspectives in relation to the aforementioned constraints, as they are able to handle hit rates of hundreds of \unit{MHz/cm^2}. And besides, being thin and integrating the front end electronics on-pixel, they reduce the material budget and the power consumption compared to hybrid pixels, which are currently the most used in tracking detectors.
ALICE, RICH, STAR-> speed limitation. LA RICHIESTA adesso è farli più veloci. 
TANTI utilizzi, possible use: also in other fields.... dosimetry. 
Finding new performing detectors is a goal not only within the framework of high energy physics, but also within that of dosimetry: in the last few years the search of radiotherapy oncological treatments with high intensity beams (FLASH mode) is requiring new dosimeters, both for the treatment and for radiological protection, as well as new beam-monitors, which are capable of deal with extreme dose rate (up to 40 \unit{Gy/s}).

I've studied the characteristics of two CMOS MAPS chips and tested them under different front end configuration. The first chip, a TJ-Monopix1 from the Monopix series, is a TowerJazz MAPS with 180 nm CMOS technology and is one of the prototypes for Belle II vertex detector upgrade expected in 2026. The second is a sensor projected and designed by ARCADIA (Advanced Readout CMOS Architectures with Depleted Integrated sensor Arrays) and is one of the Main Demonstrator-1: it is a general purpose -> dire così
 
 
 sensor which is still in a R$\&$D framework produced by LFoundry with 110 nm CMOS technology. 
Since one of the main challenges of MAPS are the differences between pixels, due to process parameters variation across the wafer, which make the sensor response nonuniform, I've performed the characterization of the chips to find the threshold and noise of the sensors and their dispersion among pixels. To make test and characterization I mainly have worked using pixels setting and testing feature available from the DAQ interface, as for example the injection circuit which allows injecting pixels with a known charge. Then, to have an absolute value in electrons of the signal recorded by TJ-Monopix1, I've also performed an absolute calibration using a Fe55 x-ray source. 

Moreover, I've also test TJ-Monopix1 at high dose rate with the FLASH-accelerator recently installed at Santa Chiara hospital in Pisa, and I have participated in the design of the setup needed for test beam measurement.
I've carried out the work in collaboration with both the Belle II and the physical medical groups in Pisa.
\end{document}

LOgicamente diverso: uno delle applicazioni è la flash.
Pixel ibridi: dirlo di più
echnological development is still a necessity-> un po' buffo
aforementioned -> no è un espressione legale, non suona bene







