\documentclass[a4paper]{report}
\usepackage[a4paper,width=150mm,top=25mm,bottom=25mm]{geometry}
\usepackage{siunitx}
\usepackage{lineno} % To obtain line numbers
\usepackage[dvipsnames]{xcolor}


\newcommand{\red}[1]{\textcolor{red}{#1}}

\begin{document}
\textbf{Characterization of monolithic CMOS pixel sensors for charged particle detectors and for high intensity dosimetry}\\


Since the 1980s, when the fabrication of device with very small electrodes (50-100 \si{\um}) became a practical possibility, pixel detectors have been widely employed for imaging and tracking charged particles in the vertex region of experiments at accelerators. Thanks to their excellent spatial resolution, today even better than 10 \si{\um}, they allow the identification of secondary vertices of short-lived particles such as $\tau$ and B mesons. 
CMOS Monolithic Active Pixel Sensors (MAPS), based on the technology of CMOS cameras, have been developed to improve the performance of the hybrid pixel detectors, in particular by reducing the amount of material, power consumption and pixel dimension.  

Experiments such as ALICE at LHC and STAR at RHIC have already introduced the CMOS MAPS technology in their detectors. ALICE Tracking System (ITS2), upgraded during the LHC long shut down in 2019-20, was the first large-area ($\sim$10 \si{m^2}) silicon vertex detector based on CMOS MAPS. Thanks to the reduction of the material budget, ITS2, which uses the ALPIDE chip developed by ALICE collaboration, obtained an amazing improvement both in the position measurement and in the momentum resolution, improving the efficiency pf track reconstruction for particle with very low transverse momentum (by a factor 6 at $p_{T}\sim$ 0.1 \si{GeV/c}).

Further advancements in CMOS MAPS technology are being aggressively pursued for the ALICE ITS3 and the Belle II vertex detector upgrades (both foreseen around 2026-27) and other experiments, with the goals of further reducing the sensor thickness and improving the readout speed of the devices, while keeping power consumption at a minimum.

Beside tracking, the development of pixel detectors is a very active field with many applications: a noteworthy example of detector originally used in particle physics, and later employed mainly for medical imaging, but also in space and for art authentication, is Medipix, a hybrid system developed at CERN within the Medipix collaboration.
Among medical applications, a possible use of CMOS MAPS could be in dosimetry: in the last few years the search of radiotherapy oncological treatments with high intensity beams (FLASH mode) is requiring new dosimeters, both for the therapies as well as new beam-monitors (especially for focused very high energy electron beams), which are capable of deal with extreme dose rate (up to 40 \si{Gy/s}).

I've studied the characteristics of two ALPIDE-like CMOS MAPS chips and tested them under different front end configuration. The first chip, the TJ-Monopix1 from the Monopix series, is a TowerJazz MAPS fabricated in 180 nm CMOS technology and is one of the prototypes for the Belle II vertex detector upgrade. The second chip, called Main Demonstrator-1, is produced by LFoundry in 110 nm CMOS technology and designed by the ARCADIA (Advanced Readout CMOS Architectures with Depleted Integrated sensor Arrays) group; it is intended to be a general purpose device with possible use in medical scanners, space experiments, future lepton colliders and also possibly X-ray applications with thick substrates.  

I have set up two test systems for the two chips in the INFN clean laboratories and characterized the devices electrically and with radioactive sources. Since one of the main challenges of MAPS are the differences between pixels, due to process parameters variation across the wafer, which make the sensor response nonuniform, I've performed the characterization of the chips to find the threshold and noise of the pixels and their dispersion across the matrix. To make test and characterization I mainly have worked using pixels setting and testing feature available from the DAQ interface, as for example the injection circuit which allows injecting pixels with a known charge. Then, to have an absolute value in electrons of the signal recorded by TJ-Monopix1, I've also performed an absolute calibration using a Fe55 x-ray source. 
Moreover, I've also test TJ-Monopix1 at high dose rate with the FLASH-accelerator recently installed at Santa Chiara hospital in Pisa, and I have participated in the design of the setup needed for test beam measurement.
I've carried out the work in collaboration with both the Belle II and the medical physics groups in Pisa.
The characterization of these test devices will be useful as input for the ongoing developments of complete devices usable in the experiments.



\end{document}








