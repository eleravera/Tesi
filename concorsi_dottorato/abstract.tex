\documentclass[a4paper]{report}
\usepackage[a4paper,width=150mm,top=25mm,bottom=25mm]{geometry}
\usepackage{siunitx}
\usepackage{lineno} % To obtain line numbers
\usepackage[dvipsnames]{xcolor}

\newcommand{\red}[1]{\textcolor{red}{#1}}

\begin{document}
\linenumbers

Since the 1980s, when the fabrication of very small electrodes (50-100 \si{\mu m}) became a possibility, pixel detectors have been widely employed for the tracking of particles in the vertex region of experiments at accelerators: thanks to their excellent spatial resolution, today even better than 10 \si{\mu m}, they allow the identification of secondary vertices of short-lived particles such as $\tau$ and B mesons. 
CMOS monolithic active pixels (MAPS), based on the technology of CMOS cameras, have been developed to improve the performance of the widely used hybrid pixel detectors, in particular by reducing the amount of material, power consumption and pixel dimension.  

\red{Due esperimenti che hanno già adottato i MAPS sono ALICE at LHC and STAR at RHIC}; ALICE Tracking System (ITS2) installed in the LHC long shut down in 2019/2020: it is the first large-area ($\sim$10 \si{m^2}) silicon vertex detector based on the CMOS MAPS. Thanks to the reduction in the material budget, compared to the preceding ITS, it led to an amazing improvement both in the point position measurement and in the momentum resolution, allowing high efficienct track recontruction of particle with very low transverse momentum (efficiency for $p_{T}\sim$ 0.1 \si{GeV/c} improved by a factor $\sim$ 6); \red{the main limitation factor of CMOS MAPS detector is the limited readout velocity , future detectors (ITS3 is planned to be installed during the long shut down three, starting in 2026) are in 
The new vertex detector (ITS3), which is planned to be installed during the long shut down starting in 2026, will be thinner and faster/Future detectors will require thin and fast cmos maps.}\\

Beside tracking, the development of pixel detectors is a very active field with many applications: some of the most notable are x-rays astrophysics and medical imaging.\\
One particular application is for dosimetry: in the last few years the search of radiotherapy oncological treatments with high intensity beams (FLASH mode) is requiring new dosimeters, both for the treatment and for radiological protection, as well as new beam-monitors, which are capable of deal with extreme dose rate (up to 40 \si{Gy/s}).

I've studied the characteristics of two CMOS MAPS chips and tested them under different front end configuration. The first chip, a TJ-Monopix1 from the Monopix series, is a TowerJazz MAPS with 180 nm CMOS technology and is one of the prototypes for Belle II vertex detector upgrade expected in 2026. The second is a sensor projected and designed by ARCADIA (Advanced Readout CMOS Architectures with Depleted Integrated sensor Arrays) and is one of the Main Demonstrator-1: it is a general purpose -> dire così
 
 
 sensor which is still in a R$\&$D framework produced by LFoundry with 110 nm CMOS technology. 
Since one of the main challenges of MAPS are the differences between pixels, due to process parameters variation across the wafer, which make the sensor response nonuniform, I've performed the characterization of the chips to find the threshold and noise of the sensors and their dispersion among pixels. To make test and characterization I mainly have worked using pixels setting and testing feature available from the DAQ interface, as for example the injection circuit which allows injecting pixels with a known charge. Then, to have an absolute value in electrons of the signal recorded by TJ-Monopix1, I've also performed an absolute calibration using a Fe55 x-ray source. 

Moreover, I've also test TJ-Monopix1 at high dose rate with the FLASH-accelerator recently installed at Santa Chiara hospital in Pisa, and I have participated in the design of the setup needed for test beam measurement.
I've carried out the work in collaboration with both the Belle II and the physical medical groups in Pisa.

LOgicamente diverso: uno delle applicazioni è la flash.
Pixel ibridi: dirlo di più
echnological development is still a necessity, un po' buffo

\end{document}








