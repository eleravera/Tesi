\label{threshold_noise}
\begin{titlepage}

\section{Charaterization of ARCADIA-MD1}


\subsection{Misure di threshold dispersion }
The signal to threshold ratio is the figure of merit for pixel detectors.\\

Il fatto di poter cambiare le soglie il time è importante in caso di problemi durante
la presa dati. Di solito chiedono di alzarle in time.\\

Invece di prendere il rapporto segnale rumore prendi il rapporto segnale soglia. Perchè?
la soglia è collegato al rumore, nel senso che: supponiamo di volere un occupancy di 10-4
allora sceglierò la soglia in base a questo. (plot su quaderno)
Da questo conto trovo la minima soglia mettibile\\
In realtà quello che faccio è mettere una soglia un po' più grande perchè il rate di rumore
dipende da molti fattori quali la temperatura, l anneling ecc, e non voglio che cambiando leggermente
uno di questi parametri vedo alzarsi molto il rate di rumore. In realtà non è solo il
rumore sensibile a diversi fattori, ma anche la soglia: ad esempio la cosa classica è
la variabilità della soglia da pixel a pixel.\\
In questo modo rumore e soglia diventano parenti.\\
Review pag 26.

The noise requirement can be expressed as:
\begin{equation}
ENC \minto \sqrt{T^{2} /3.7 - T_{RMS}(t, x)}
\end{equation}

Questo implica tra le altre cose che voglio poter assegnare delle soglie diverse
a diversi pixel: Drawback è dare spazio per registri e quantaltro.\\
Questo lascia però ancora aperto il problema temporale delle variazioni del rumore:
problema per cui diventano necessarie le misure dei sensori dopo l'irraggiamento.\\


Per arcadia i registri (c'è un DAC) per la soglia (VCASN) si trovano in periferia.
Non fare trimming sulla soglia è uno dei problemi che si sono sempre incontrati: a casusa dei mismatch dei transistor
le soglie efficaci pixel per pixel cambiano tanto.
La larghezza della s curve è il noise se assumi che il noise è gaussiano

\subsection{Com'è fatto il set up per le misure}
Com'è fatto il set up per le misure.\\
FPGA BB, Chip con FE board, qualche foto\\
\end{titlepage}
