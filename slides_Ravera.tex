%%%%%%%%%%%%%%%%%%%%%%%%%%%%%%%%%%%%%%%%%%%%%%%%%%%%
%
% Copyright 2022 
%
% This file may be distributed and/or modified
%
% 1. under the LaTeX Project Public License and/or
% 2. under the GNU Public License.
%
%%%%%%%%%%%%%%%%%%%%%%%%%%%%%%%%%%%%%%%%%%%%%%%%%%%%%%%%%%%

%%%   DESCRIZIONE 

% Non riuscendo a trovare un tema per la classe Beamer che mi soddisfacesse, 
% ho provveduto alla creazione del seguente tema che ho utilizzato per la realizzazione 
% della presentazione che ha accompagnato la discussione di tesi di laurea. 
% Il tema è basato a partire dal colore blu del logo dell'università degli
% studi di Trieste, dal quale è stata creata una palette di colori aggiungendo 
% gradualmente del bianco. 
% 
%  Nota:  
% Per il caso specifico (discussione di tesi) in cui mi è servito il tema, ho impostato ad hoc
% una disposizione alternativa del primo frame rispetto a quello che si otterrebbe normalmente 
% con il comando \maketitle. 
% Per evitare questo basta commentare il comando \setTitlestyleDissertation immediatamente precedente. 
%
% Il layout proposto potrebbe essere leggermente rivisto, per es. nel caso in cui fosse 
% presente anche un correlatore. In questo bisogna decommentare nel file:
%		otherResources/presentazione_myThemeSetting.tex
% le opportune linee di codice dove suggerito. 
% Sempre agendo nel suddetto file è possibile anche impostare il layout utilizzando solamente 
% il logo dell'università, omettendo quindi il logo del dipartimento. In generale da tale 
% file è possibile apportare qualisasi modifica al tema. Consiglio di consultare la 
% documentazione del paccehtto beamer all'indirizzo: 
%	https://mirror.math.princeton.edu/pub/CTAN/macros/latex/contrib/beamer/doc/beameruserguide.pdf

\documentclass{beamer}

% Caricamento di tutti i pacchetti necessari 
\usepackage{presentation/setting/packages}
%\setbeamersize{text margin left=5mm,text margin right=5mm} 


% Settaggi documento %%
        
\hypersetup{
    pdfauthor={Nome Cognome},
    pdftitle={Titolo presentazione},
    pdfsubject={Argomento presentazione}
}
\title[Titolo breve]{Characterization of monolithic CMOS pixel sensors for charged particle detectors and for high intensity dosimetry}
%\subtitle{Il difficile ruolo delle navi della federazione ai confini della zona di spazio neutrale}
\institute{Università degli studi di Pisa}
\author[E. Ravera]{Eleonora Ravera}
\year=2022
\month=10
\day=24
\def\relatore{Francesco Forti}
\def\relatoreLabel{Supervisor:}
\def\correlatore{Maria Giuseppina Bisogni}
\def\correlatoreLabel{Co-supervisor:}
\def\candidatoLabel{Candidate:}
\def\LogoUniversita{figures/Logo-UNIPisa.png}
\def\LogoDipartimento{figures/Logo-UNIPisa.png}
%\def\LogoFiligrana{otherResources/background-blu.png}

%Applico tutte le personalizzazioni desiderate al tema: 
\input{presentation/setting/theme_settings.tex}

% Inizio Presentazione

\begin{document}
	 
		%%%%%%%%%%%%%%%%%%%%%%%
		%%  Slide 1: TITOLO  %%
		%%%%%%%%%%%%%%%%%%%%%%%
\begin{frame}
\setTitlestyleDissertation
\maketitle
\end{frame}

\usetikzlibrary{decorations.pathreplacing}        

\definecolor{barcolor}{RGB}{160,196,222}
\setbeamercolor{lightbluebox}{bg=barcolor}


\section{MAPS}

    %%%%%%%%%%%%%%%%%%%%%%%%%%%%%%%%%%%%%%%%
    %%  Slide 1: <Pixels types>  %%
    %%%%%%%%%%%%%%%%%%%%%%%%%%%%%%%%%%%%%%%%
    \begin{frame}
        \frametitle{Pixel detectors}
        \begin{itemize}
            \item caratteristiche standard e vantaggi rispetto agli ibridi?
        \end{itemize}
        3 strade diverse: MAPS, hybrid e CCDs.
    \end{frame} 


    %%%%%%%%%%%%%%%%%%%%%%%%%%%%%%%%%%%%%%%%
    %%  Slide 2: <CMOS MAPS concept>  %%
    %%%%%%%%%%%%%%%%%%%%%%%%%%%%%%%%%%%%%%%%
    \begin{frame}
        \frametitle{CMOS Monolitich Active Pixel Sensors}
        %forse anche la terza parentesi con minore circa 100 um
        \begin{columns}
            \column{0.4\textwidth}
                \includegraphics[width=1.1\linewidth]{figures/Pixel_detectors/MAPS_scheme.png}
            \column{0.6\textwidth}  
                \begin{tikzpicture}[overlay]
                    \draw[decorate,decoration={brace}]
                        (0,-0.4) -- node[xshift=2pt,anchor=west] {$\lesssim$\SI{5}{\um} electronics}(0,-0.5);
                \medskip
                    \draw[decorate,decoration={brace}]
                        (0,-1) -- node[xshift=2pt,anchor=west] {$\lesssim$\SI{50}{\um} sensor}(0,-1.7);
                \end{tikzpicture}
                \bigskip
                \begin{equation*}
                    \hspace{80pt} d \propto \sqrt{\rho V}
                \end{equation*}   
                \begin{equation*}
                    %\hspace{75pt} ENC^2 \propto \frac{4}{3} \frac{kT}{g_m} \frac{C_D ^2}{\tau_{sh}}
                    \hspace{80pt} ENC^2 \propto C_D ^2
                \end{equation*}  
                \begin{equation*}
                    %\hspace{75pt} \tau \propto \frac{1}{g_m}\frac{C_D}{C_f}
                    \hspace{80pt} \tau \propto C_D
                \end{equation*}  
                \begin{equation*}
                    \hspace{80pt} Q_{MIP} \sim 80 e^-/\si{\um}
                \end{equation*}  
            \end{columns}   
        \medskip 
        \begin{itemize}
            \item Electronics is low resistivity $\rho$ while sensor needs high $\rho$, special technologies for the sensor. Next slide!  High resistivity allows for depleted epi-layer DMAPS
            \item If not completed depleted charge is collected by diffusion
            \item Very low capacity $\sim$\SI{1}{fF/cm\squared}
            \item epitaxial layer with doping few order of magnitude smaller than the subtrate
        \end{itemize}
    \end{frame} 



    %%%%%%%%%%%%%%%%%%%%%%%%%%%%%%%%%%%%%%%%
    %%  Slide 3: <Sensors types>  %%
    %%%%%%%%%%%%%%%%%%%%%%%%%%%%%%%%%%%%%%%%
    \begin{frame}
        \frametitle{Sensor types}
            \begin{itemize}
                \item Large and small fill factor, depending on the deep p-well structures
            \end{itemize}
            $<$\SI{5}{fF} vs $\sim$100-200\si{fF}, formule ENC e tau

            \centering
            \includegraphics[width=.9\linewidth]{figures/Pixel_detectors/large_small_sensor_scheme.png}\\
            \begin{columns}
                \column{0.5\textwidth}  
                    \centering
                    \includegraphics[width=0.8\linewidth]{figures/Pixel_detectors/ALPIDE_after_PM.png}
                \column{0.5\textwidth}  
                    \begin{itemize}
                        \item Process modification with a low dose planar implant \\
                        main investigator ALICE\\
                        no need of change in the sensor and circuit layout
                    \end{itemize} 
            \end{columns}

            \end{frame} 

    %%%%%%%%%%%%%%%%%%%%%%%%%%%%%%%%%%%%%%%%
    %%  Slide 1: <READOUT>  %%
    %%%%%%%%%%%%%%%%%%%%%%%%%%%%%%%%%%%%%%%%
    \begin{frame}
        \frametitle{Front end}
            \begin{columns}
                \column{0.75\textwidth}  
                Pixel area economy and dimension of components are extremely relevant. MAPS usage allowed by miniaturization of components: starting from \SI{600}{nm} down to 120-180{nm} CMOS imaging process of transistors
                \column{0.35\textwidth}  
                    \begin{figure}[h!]
                        \centering
                        \includegraphics[width=0.99\linewidth]{figures/Monopix1/Monopix1_2x2pixelsgroup.png}
                    \end{figure}
            \end{columns}


        \begin{itemize}
            \item Analog or digital output: ToT is a compromise
            \item Rolling shut or sparsified (data push) readout 
            \item Triggered or triggerless
            \item Column drain is one of the most popular readout mechanism, but other possible are possible
            \item Buffer on pixel to store more than one hit data
        \end{itemize}
 
    \end{frame} 
\section{Applications}

    %%%%%%%%%%%%%%%%%%%%%%%%%%%%%%%%%%%%%%%%
    %%  Slide 1: <ALICE>  %%
    %%%%%%%%%%%%%%%%%%%%%%%%%%%%%%%%%%%%%%%%
    \begin{frame}
        \frametitle{ALICE}
        \begin{figure}[h!]
            \centering
            \includegraphics[width=.45\linewidth]{figures/pixel_detectors_usage/alice.png}
            \includegraphics[width=.45\linewidth]{figures/pixel_detectors_usage/ALICE_FoCAL.png}
        \end{figure}
    \end{frame} 



    %%%%%%%%%%%%%%%%%%%%%%%%%%%%%%%%%%%%%%%%
    %%  Slide 1: <>  %%
    %%%%%%%%%%%%%%%%%%%%%%%%%%%%%%%%%%%%%%%%
    \begin{frame}
        \frametitle{FLASH radiotherapy}
        \begin{figure}[h!]
            \centering
            \includegraphics[width=.8\linewidth]{figures/pixel_detectors_usage/curve_flash.png}
        \end{figure}
    \end{frame} 
\section{TJ-Monopix1}
    %%%%%%%%%%%%%%%%%%%%%%%%%%%%%%%%%%%%%%%%
    %%  Slide 1: <READOUT>  %%
    %%%%%%%%%%%%%%%%%%%%%%%%%%%%%%%%%%%%%%%%
    \begin{frame}[noframenumbering]
        \frametitle{Front end electronics}
            \begin{columns}
                \column{0.7\textwidth}  
                Pitch$\sim$\SI{50}{\um} $\rightarrow$ \textbf{pixel area economy} and dimension of components are extremely relevant. \\\smallskip
                %MAPS usage allowed by miniaturization of components, i.e. TJ-Monopix1 is 180{nm} CMOS process.\\
                \column{0.4\textwidth}  
                    \begin{figure}[h!]
                        \vspace*{-0.9cm}\hspace*{-0.9cm}
                        \includegraphics[width=1.06\linewidth]{figures/Monopix1/Monopix1_2x2pixelsgroup.png}
                    \end{figure}
            \end{columns}

            \begin{columns}
                \column{0.45\textwidth} 
                    The pixel area include the:
                    \begin{itemize}
                        \item Analog front end
                        \item Digital readout 
                    \end{itemize} 
                \column{0.55\textwidth} 
                    \vspace*{-0.15cm}%\hspace*{+0.15}
                    \begin{beamercolorbox}[rounded=true, center]{palette light primary}
                        \setlength{\tabcolsep}{0.5em} % for the horizontal padding
                        {\renewcommand{\arraystretch}{1.2}% for the vertical padding
                        \begin{tabular}{l|l}
                            \circled{Analog} & Digital\\
                            \hline
                            Triggered & \circled{Triggerless}\\
                            \hline
                            Buffer & \circled{No buffer} \\
                            \hline
                            Rolling shutter & \circled{Sparsified}\\
                        \end{tabular}
                        }
                    \end{beamercolorbox}
            \end{columns}
    \end{frame} 



    %%%%%%%%%%%%%%%%%%%%%%%%%%%%%%%%%%%%%%%%
    %%  Slide 1: <TJ-Monopix1>  %%
    %%%%%%%%%%%%%%%%%%%%%%%%%%%%%%%%%%%%%%%%
    \begin{frame}
        \frametitle{TowerJazz Monopix1}
        \begin{itemize}
            \item Designed by ATLAS collaboration
            \item Produced by TowerJazz, an electronic foundry located in Israel
            \item Small electrode design: \SI{2}{\um} with C=\SI{3}{fF}
            \item The sensor implements a process modification in the epi-layer that allows the creation of a planar junction (ALICE)
            \item The Front End has 4 flavors, all ALPIDE-like (ALPIDE chip design by ALICE and used in the ITS2)
        \end{itemize}
        \begin{columns}
            \column{0.5\textwidth}  
                \begin{table}
                    \begin{tabular}{| c |c |}
                    \hline
                    Resistivity & $>$\SI{1}{k\ohm cm}\\
                    Matrix size &  1$\times$2\si{cm\squared}\\
                    Pixel size & 36 $\times$ 40 \si{\um\squared}\\
                    Max depletion & \SI{25}{\um}\\
                    Time resolution & \SI{25}{ns} \\
                    %Power cons. & $\sim$ 120 \si{mW/cm\squared}\\    
                    \hline
                    \end{tabular}
                \end{table}
            \column{0.5\textwidth} 
                \hspace*{-0.4cm}\includegraphics[width=1.1\linewidth]{figures/Monopix1/Monopix1_section_scheme_ngap.png}\\
        \end{columns}
    \end{frame} 


    %%%%%%%%%%%%%%%%%%%%%%%%%%%%%%%%%%%%%%%%
    %%  Slide 1: <TJ-Monopix1>  %%
    %%%%%%%%%%%%%%%%%%%%%%%%%%%%%%%%%%%%%%%%
    \begin{frame}
        \frametitle{TJ-Monopix1 analog output}
        \begin{columns}
            \column{0.5\textwidth}
            ToT = \textbf{Time Over Threshold}\\
            \bigskip
            \hspace*{-0.5cm} 
                \includegraphics[width=1.25\linewidth]{figures/Monopix1/tot_example.pdf}
            \column{0.5\textwidth}  
                \begin{itemize}
                    \item The reset circuit of the amplifier made with a \textbf{PMOS}, guaranties the discharge of the preamplifier with constant slope
                    \item The ToT grows linearly with the pulse amplitude
                    \item Before readout ToT is stored in RAM as a 6-bits variable $\rightarrow$ then the ToT can vary in range 0-\SI{1.6}{\us}
                \end{itemize}
        \end{columns}
    \end{frame} 



 








\section{Charaterization}
    %%%%%%%%%%%%%%%%%%%%%%%%%%%%%%%%%%%%%
	%%  Slide 1: <> %%
	%%%%%%%%%%%%%%%%%%%%%%%%%%%%%%%%%%%%%
    \begin{frame}
        \frametitle{Threshold and noise}
        The threshold and the noise are \textbf{strictly} related with the \textbf{FE} status, and in particular to the threshold of the \textbf{discriminator}.\\
        \medskip
        The lower the threshold, the lower the minium detectable signal, allowing for thinner detector (Q$_{MIP}\sim$\SI{80}{e/\um}). \\
        \medskip
        What determins the minimum stable threshold?
        \begin{itemize}
            \item the \textbf{ENC} 
            \item the \textbf{threshold dispersion} 
            \item threshold variation in \textbf{time}
        \end{itemize}
        \medskip
        \centering\begin{beamercolorbox}[sep=0em,wd=0.85\textwidth,ht=1.5ex, dp=0.1ex, rounded=true, center]{lightbluebox}
            noise \SI{9}{\elementarycharge}$^-$    threshold \SI{270}{\elementarycharge}$^-$    threshold dispersion\SI{30}{\elementarycharge}$^-$
        \end{beamercolorbox}
    \end{frame}


    %%%%%%%%%%%%%%%%%%%%%%%%%%%%%%%%%%%%%
	%%  Slide 1: <Scurve> %%
	%%%%%%%%%%%%%%%%%%%%%%%%%%%%%%%%%%%%%
    \begin{frame}
        \frametitle{Threshold and noise: how measure them?}
        \textbf{Injection circuit}: allows injecting a fixed charge Q on a capacity (C$_{inj}$=\SI{230}{fF}) at the FE input
        \bigskip
        \begin{columns}
            \column{0.45\textwidth}           
                  \includegraphics[width=1.1\linewidth]{figures/charaterization/scurve.pdf}
            \column{0.55\textwidth} 
                %\begin{center}
                \begin{beamercolorbox}[sep=0em,wd=0.45\textwidth,ht=1.5ex, dp=0.1ex, rounded=true, center]{lightbluebox}
                    F=\SI{20}{\elementarycharge}$^-$/DAC
                \end{beamercolorbox}
                %\end{center}                    
                \\
                \bigskip
                Assuming a gaussian noise:
                \begin{itemize}
                    \item the threshold is the 50\%
                    \item the noise is 1/slope 
                \end{itemize}
                \medskip
                Analitical parametrization of the curve with $error\,function$
                %\begin{equation*}
                %    \tiny
                %    f(x, \mu, \sigma) = \frac{1}{2} \; \left(1\,+\,erf\left(\frac{x-\mu}{\sigma \sqrt{2}}\right)\right)
                %    \label{eq:fit_scurve}
                %\end{equation*}
        \end{columns}
    \end{frame}


    %%%%%%%%%%%%%%%%%%%%%%%%%%%%%%%%%%%%%%%%
    %%  Slide 2: <Threshold_noise>  %%
    %%%%%%%%%%%%%%%%%%%%%%%%%%%%%%%%%%%%%%%%
    \begin{frame}
        \frametitle{Threshold and noise}

            \begin{figure}
                \includegraphics[width=.45\linewidth]{figures/charaterization/threshold_map.pdf}
                \includegraphics[width=.45\linewidth]{figures/charaterization/noise_map.pdf}                
            \end{figure}

        Injection circuit broken.\\
        With this threshold the mean rate is $\sim$\SI{3}{Hz} on all the matrix, excluding the noisy pixels!
        Changing the register which sets the discriminator threshold the effective threshold changes in range (370-500). NEed for more bits and for a fine tuning per pixel
    \end{frame}

    %%%%%%%%%%%%%%%%%%%%%%%%%%%%%%%%%%%%%%%%
    %%  Slide 4: <ToT vs charge>  %%
    %%%%%%%%%%%%%%%%%%%%%%%%%%%%%%%%%%%%%%%%
    \begin{frame}
        \frametitle{Calibration}
        Q injected in DAC depends on the C$_{inj}$, different for each pixel.     
        \bigskip
        How convert the DAC units in electrons collected in the sensor?\\
        \begin{itemize}
            \item take advantage linearity of the output signal 
            \item need of a reference, radioactive source
        \end{itemize}
        \medskip
        \begin{beamercolorbox}[sep=0em,wd=0.85\textwidth,ht=1.5ex, dp=0.1ex, rounded=true, center]{lightbluebox}
            Fe$^{55}$ $\rightarrow$ Mn$^{55}$ + K$_\alpha$ (\SI{5.9}{keV}) or K$_\beta$ (\SI{6.5}{keV})
        \end{beamercolorbox}
        \begin{tikzpicture}[overlay]
            \draw[decorate,decoration={brace}]
                (3.5,0) -- (5.5,0);
        \end{tikzpicture}
        \begin{columns}
            \column{0.2\textwidth}             
            \column{0.8\textwidth}             
            \begin{itemize}
                \item $\lambda$=\SI{29}{\um}$\rightarrow$ P$_{abs}$=
                \item $w_i$=\SI{3.6}{eV} in Si @ \SI{300}{\kelvin}$\rightarrow$ \SI{1616}{\elementarycharge}$^-$
            \end{itemize}
        \end{columns}
    \end{frame}  


    %%%%%%%%%%%%%%%%%%%%%%%%%%%%%%%%%%%%%%%%
    %%  Slide 3: <ToT linearity>  %%
    %%%%%%%%%%%%%%%%%%%%%%%%%%%%%%%%%%%%%%%%
    \begin{frame}
        \frametitle{ToT linearity}
        \begin{itemize}
            \item linearity is due to the \textbf{PMOS} reset circuit of the amplifier which guaranties that the discharge of the preamplifier is costant in time. Then the ToT is completely correlated with the pulse amplitude 
            \item ToT is saved as a 6-bits variable and can ToT varies in range 0-63
        \end{itemize}
        \medskip
        \medskip
        \medskip
        \begin{columns}
            \column{0.6\textwidth}  
                \centering
                Linearity tested with the injection
                \begin{figure}[h!]
                    \includegraphics[width=.49\linewidth]{figures/charaterization/ToT_rollover.png}
                    \includegraphics[width=.49\linewidth]{figures/charaterization/ToT_rollover.png} 
                \end{figure}
            \column{0.4\textwidth}
                \centering\textbf{Rollover!}\\
                \smallskip
                \includegraphics[width=.99\linewidth]{figures/Monopix1/rollover.pdf} 
        \end{columns}
    \end{frame}    
      

    %%%%%%%%%%%%%%%%%%%%%%%%%%%%%%%%%%%%%%%%
    %%  Slide 4: <ToT calibration>  %%
    %%%%%%%%%%%%%%%%%%%%%%%%%%%%%%%%%%%%%%%%
    \begin{frame}
        \frametitle{ToT absolute calibration}
        \begin{columns}
            \column{0.4\textwidth}                  
                \includegraphics[width=1.1\linewidth]{figures/charaterization/fit_line_gauss_r69.pdf}
            \column{0.4\textwidth}    
            \begin{itemize}
                \item fit to find the peak position in the spectrum
                \item conversion from ToT in DAC (using the line parameters)
                \item peak value in DAC fixed to be equal to \SI{1616}{\elementarycharge}$^-$
            \end{itemize} 
        \end{columns}
        
    \end{frame}     




    %%%%%%%%%%%%%%%%%%%%%%%%%%%%%%%%%%%%%%%%
    %%  Slide 4: <Aquisitions with sources>  %%
    %%%%%%%%%%%%%%%%%%%%%%%%%%%%%%%%%%%%%%%%
    \begin{frame}
        \frametitle{Acquisition with sources}
        \begin{columns}
            \column{0.4\textwidth}                  
                \includegraphics[width=.99\linewidth]{figures/charaterization/Sr90_spectrum_cluster.pdf}
            \column{0.4\textwidth}                  
        \end{columns}
        
    \end{frame}    


    %%%%%%%%%%%%%%%%%%%%%%%%%%%%%%%%%%%%%%%%
    %%  Slide 5: <ToT bias>  %%
    %%%%%%%%%%%%%%%%%%%%%%%%%%%%%%%%%%%%%%%%
    \begin{frame}
        \frametitle{Changing the bias}
        \begin{columns}
            \column{0.3\textwidth} 
                Acquisitions with:
                \begin{itemize}
                    \item Injection
                    \item Fe$^{55}$ source
                \end{itemize}
            \column{0.7\textwidth} 
                %Maximum bias suggested
                \begin{table}
                    \begin{center}
                    \scalebox{0.75}{
                    \begin{tabular}{| c |  c | c | c |}
                    \hline
                    & -\SI{6}{V} & -\SI{3}{V} & \SI{0}{V}\\
                    \hline
                    \hline
                    Threshold [DAC] & 20 $\pm$ 2 & 21 $\pm$ 2 & 24 $\pm$ 2\\
                    Noise [DAC] & 0.61 $\pm$ 0.08 & 0.62 $\pm$ 0.08 & 0.82 $\pm$ 0.1\\
                    \hline
                    \end{tabular}}
                    \end{center}
                \end{table}
        \end{columns}
 
        \begin{columns}
            \column{0.4\textwidth}          
            \includegraphics[width=1.2\linewidth]{figures/charaterization/Fe_spectrum_bias.pdf}
        \column{0.6\textwidth}
            Reducing the bias from -\SI{6}{V} to \SI{0}{V} eduction of the below quantity in the Fe$^{55}$ spectrum: 
            \begin{itemize}
                \item ToT value of the peak $\sim$30\%
                \item N of events under the peak $\sim$60\%
                \item hit rate $\sim$60\%
            \end{itemize}
          
        \end{columns}
    \end{frame}      

    %%%%%%%%%%%%%%%%%%%%%%%%%%%%%%%%%%%%%%%%
    %%  Slide 5: <Dead time>  %%
    %%%%%%%%%%%%%%%%%%%%%%%%%%%%%%%%%%%%%%%%
    \begin{frame}
        \frametitle{Readout time}
        \textbf{Injection} allows injecting pulses at different rate. 
        \begin{itemize}
            \item no memory on pixel: 
            \item readout is completely sequential: one serializer @ \SI{40}{MHz}
            \item each hit is a 27-bits data packet, at least \SI{675}{ns} needed
        \end{itemize}

        \medskip
        \begin{columns}
            \column{0.55\textwidth}              
                \includegraphics[width=1.07\linewidth]{figures/charaterization/default_line.pdf}
            \column{0.45\textwidth}  
            Readout time \textbf{slightly} depends on the FE status and can be reduced down to \SI{31}{clk.}cnts = \SI{775}{ns} per pixels 

        \end{columns}            
    \end{frame}          
\section{CS.Chiara}
        
    %%%%%%%%%%%%%%%%%%%%%%%%%%%%%%%%%%%%
    %% Slide 1: <> %%
    %%%%%%%%%%%%%%%%%%%%%%%%%%%%%%%%%%%%
    \begin{frame}
        \frametitle{ElectronFlash}
        \begin{figure}[h!]
            \centering
            \includegraphics[width=.2\linewidth]{figures/test_beam/beam_structure.pdf}
        \end{figure}
    \end{frame}

    %%%%%%%%%%%%%%%%%%%%%%%%%%%%%%%%%%%%
    %% Slide 2: <> %%
    %%%%%%%%%%%%%%%%%%%%%%%%%%%%%%%%%%%%
    \begin{frame}
        \frametitle{ElectronFlash}
        \begin{figure}[h!]
            \centering
            \includegraphics[width=.2\linewidth]{figures/test_beam/Flash-beam-scheme.pdf}\\
            \includegraphics[width=.2\linewidth]{figures/test_beam/carrello.jpeg}       
        \end{figure}
    \end{frame}    

    
    %%%%%%%%%%%%%%%%%%%%%%%%%%%%%%%%%%%%
    %% Slide 3: <> %%
    %%%%%%%%%%%%%%%%%%%%%%%%%%%%%%%%%%%%
    \begin{frame}
        \frametitle{ElectronFlash}
        \begin{figure}[h!]
            \centering
            \includegraphics[width=.2\linewidth]{figures/test_beam/Flash-beam-scheme.pdf}\\     
        \end{figure}
    \end{frame}   
\input{presentation/conclusions.tex}

%    %%%%%%%%%%%%%%%%%%%%%%%%%%%%%%%%%%%%%%%%
    %%  Slide 1: <BACKUP>  %%
    %%%%%%%%%%%%%%%%%%%%%%%%%%%%%%%%%%%%%%%%
    \begin{frame}
        Backup        
    \end{frame} 



    %%%%%%%%%%%%%%%%%%%%%%%%%%%%%%%%%%%%
    %% Slide 3: <> %%
    %%%%%%%%%%%%%%%%%%%%%%%%%%%%%%%%%%%%
    \begin{frame}
        \frametitle{Test on beam: preliminary results}
        \begin{itemize}
            \item With \textbf{both} the collimators, DPP=\SI{0.07}{Gy}, t$_p$=\SI{4}{\us}, PRF=\SI{1}{Hz}
        \end{itemize}
        \medskip
        \begin{figure}
            \includegraphics[width=0.49\linewidth]{figures/test_beam/Q1_17_11.pdf}
            \includegraphics[width=.49\linewidth]{figures/test_beam/Q2_17_11.pdf}
        \end{figure}
        \begin{itemize}
            \item the collimators do not shield the detectors by all particles
            \item probably photons are produced by electrons in Al collimators
            \item for each accelerator pulse, 2 readout "cycle"  
        \end{itemize}
    \end{frame} 
    
    %%%%%%%%%%%%%%%%%%%%%%%%%%%%%%%%%%%%
    % Slide 3: <> %%
    %%%%%%%%%%%%%%%%%%%%%%%%%%%%%%%%%%%%
    \begin{frame}
        \frametitle{Test on beam: preliminary results}
        \begin{itemize}
            \item\textbf{Without} any collimator, DPP=\SI{0.04}{Gy}, t$_p$=\SI{4}{\us}, PRF=\SI{1}{Hz}
            \item MIP are expected to release \SI{2000}{\elementarycharge}$^-$, and because of rollorver are expected to be 300-400\si{\elementarycharge}$^-$
        \end{itemize}
        \medskip
        \begin{columns}
            \column{0.5\textwidth} 
                \includegraphics[width=1.1\linewidth]{figures/test_beam/Qe_17_32.pdf}
            \column{0.5\textwidth} 
                \begin{itemize}
                    \item ToT converted in charge
                    \item pixels turn on in N clock counts
                    \item after each pulse an induced signal on the whole matrix
                \end{itemize}
        \end{columns}   
        \medskip
        Need for a simulation to understand the data
    \end{frame}   

        
\end{document}